\documentclass[12pt, letterpaper]{article}
\usepackage[utf8]{inputenc}
\usepackage{enumitem}
\usepackage{tikz}
\usepackage{amsmath}

\def\checkmark{\tikz\fill[scale=0.4](0,.35) -- (.25,0) -- (1,.7) -- (.25,.15) -- cycle;} 

\title{ECG 782 Homework \#1}
\author{Minsung Cho}
\date{2021/01/20}

\begin{document}

\maketitle

\begin{enumerate}
\item Getting started
	\begin{enumerate}
	\item Determine how you will use \LaTeX{}:
		\begin{itemize}
		\item I will be using TeXworks.
		\end{itemize}
	\item Download the ``standard'' test images from the Gonzalez and Woods website. \checkmark
	\item Download the sample images fro the class website. \checkmark
	\item Indicate the method you have selected for \LaTeX{} use. \checkmark
	\item Generate your report using the article class. \checkmark
	\end{enumerate}
\item Histogram Equalization
	\begin{enumerate}
	\item Write a function hist\_eq.m that performs histogram equalization on an intensity image. The function should take as inputs an intensity image and the number of gray level value bins. Create a separate m-file for this function.
	\item Perform histogram equalization on the jetplane image using 256, 128, and 64 bins. Compare the original image and the histogram equalized images by plotting the corresponding histograms and images side-by-side in a 2 × 2 subplot matrix.
	\item Perform the equalization in 32 x 32 blocks. Display the output image. You will find blockproc.m useful.
	\end{enumerate}
\item Basic Morphology
	\begin{enumerate}
	\item Threshold the image SJEarthquakesteampic.jpg to detect faces. Be sure to describe
how you obtained your threshold. You may find this is easier in another colorspace such
as HSV.
	\item Use morphological operations to clean the image. Count the number of players in the
cleaned threshold image.
	\item  Create an output image that has a bounding box around each face. Use regionprops.m.
In your report, create an output figure with three images in a row. (a) is the face threshold image, (b) morphologically cleaned image, and (c) the color image with bounding
box around face areas.
	\item Repeat for barcelona-team.jpg. Explain the differences you found.
	\end{enumerate}
\item Filtering
	\begin{enumerate}
	\item Consider image DSCN0479-001.JPG as a perfect image. Add white Gaussian noise with
variance 0.005. Smooth with a 3 × 3 and 7 × 7 box filter and a median filter. Compute
the mean squared error (MSE)
	\begin{align*}
	MSE=\frac{1}{MN} \sum_{m} \sum_{n} (I_1(m,n)-I_2(m,n))^2
	\end{align*}
	and the peak signal-to-noise ratio (PSNR)
	\begin{align*}
	PSNR=20\times log_{10}(255/\sqrt{MSE})
	\end{align*}
	for the noise reduced images. Compile results using a \LaTeX{} Table. Which filter has the best results based on the error measures? How do the results compare visually?
	\item Repeat (a) with salt and pepper noise with noise density 0.05. Compile results using a \LaTeX{} Table.
	\item Do the filtering again but this time on a real noisy image DSCN0482-001.JPG obtained at higher ISO. Compare the results visually only this time. Which filter works best for ``real'' noise? How much time does each type of filter require (use tick.m and toc.m)?
	\end{enumerate}
\end{enumerate}



\end{document}